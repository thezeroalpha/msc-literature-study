% vim: spell spelllang=en_us
Embedded devices are a common, sometimes invisible, part of our lives.
Like any other device, they can have security vulnerabilities.
The implications of these vulnerabilities may be more serious and immediate than on desktop or mobile systems, as embedded devices are frequently used in critical applications, such as the medical and industrial sectors.
In this paper, we conduct a systematic study of past analyses of embedded device firmware.
We find that most analyses are small-scale in terms of the number of firmware samples, and those that are on a larger scale tend to focus on Linux-based firmware.
Still, a large number of embedded devices use non-Linux-based firmware, so this group should not be neglected.
Our results indicate that a large-scale study of non-Linux-based firmware, along with the development of generic analysis techniques for unknown firmware, could be a promising direction for future research.
