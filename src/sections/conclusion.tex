% vim: spell spelllang=en_us
\section{Conclusion}\label{s:conclusion}
In this paper, we discussed existing research in the area of embedded device firmware analysis.
We proposed a classification method where we separated studies by whether they analyzed only Linux-based firmware, only non-Linux-based firmware, or both.
We then categorized them based on several other factors, such as the analysis methods they used and the number of firmware samples they analyzed.
We found that the majority of large-scale studies focused mainly on Linux-based firmware, with a small number analyzing both.
Most non-Linux-based studies examine firmware built for the ARM architecture, with prior knowledge of the target device specifications, and the most common device sector for both Linux-based and non-Linux-based firmware is the \q{Internet of Things}.
The results indicate that development of scalable generic analysis techniques for unknown non-Linux firmware could be a useful direction for future work.
