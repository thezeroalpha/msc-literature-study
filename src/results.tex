\section{Results}
In total, we collected \result{total-number-of-papers} relevant sources of information.
The corpus contains literature discussing both LB and NLB firmware.
Of these \result{total-number-of-papers} texts, \result{total-linux-only} focus only on LB firmware, \result{total-nonlinux-only} only on NLB firmware, and \result{total-both-only} only on \textit{both} LB and NLB firmware.
Hence, there are \result{total-not-only-linux} publications that do not focus only on LB firmware; they are our main point of focus for this study.

how does the landscape of firmware analysis look?

what large scale studies have there been?
how has non-Linux firmware been analyzed?

% Code listings are available, e.g. \autoref{code:sample}.

% \begin{lstlisting}[float=tb,caption={Some caption.}, label={code:sample}, language=C]
% #include <stdio.h>
% int main() {
%   printf("Hello World\n");
% }
% \end{lstlisting}

% I can use subtables, like \autoref{tab:bigtable}.

% \begin{table}[tb]
%   \begin{subtable}[t]{\textwidth}
%     \centering
%     \begin{tabular}{l r | r }
%                       & \multicolumn{2}{c}{\underline{Create}} \\
%        Heading & Heading                         & Heading \\
%       \hline \hline
%       Heading           & \result{sample}                  & \result{sample}      \\
%     \end{tabular}
%     \caption{Some caption.}
%     \label{tab:subtable 1}
%   \end{subtable}

%   \bigskip

%   \begin{subtable}[t]{\textwidth}
%     \centering
%     \begin{tabular}{l r | r }
%                       & \multicolumn{2}{c}{\underline{Create}} \\
%        Heading & Heading                         & Heading \\
%       \hline \hline
%       Heading           & \result{sample}                  & \result{sample}      \\
%     \end{tabular}
%     \caption{Some caption.}
%     \label{tab:subtable 2}
%   \end{subtable}

%   \bigskip

%   \begin{subtable}[t]{\textwidth}
%     \centering
%     \begin{tabular}{l r | r }
%                       & \multicolumn{2}{c}{\underline{Create}} \\
%        Heading & Heading                         & Heading \\
%       \hline \hline
%       Heading           & \result{sample}                  & \result{sample}      \\
%     \end{tabular}
%     \caption{Some caption.}
%     \label{tab:subtable 3}
%   \end{subtable}

%   \caption{It's a big table}
%   \label{tab:bigtable}
% \end{table}

% Or simple tables, like \autoref{tab:smalltable}.

% \begin{table}[tb]
%   \centering
%   \begin{tabular}{l | r | r}
%     Heading & Heading                         & Heading                         \\
%     \hline \hline
%     Row       & \result{sample}    & \result{sample} \\
%     Row     & \result{sample} & \result{sample}
%   \end{tabular}
%   \caption{Smaller table}
%   \label{tab:smalltable}
% \end{table}
% \clearpage
